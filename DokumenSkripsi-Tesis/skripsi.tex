%% skripsi.tex
%% Copyright 2011-2020 Lionov
%%
%% This file is part of the 'Template Skripsi/Tugas Akhir FT UNPAR'
%% A project forked from 'Template Skripsi/Tugas Akhir FTIS UNPAR'
%%
%% This work may be distributed under the conditions of 
%% the LaTeX Project Public License, either version 1.3 of this license or
%% any later version. The latest version of this license is in
%%   http://www.latex-project.org/lppl.txt
%% and version 1.3 or later is part of all distributions of LaTeX
%% version 2005/12/01 or later.
%%
%% This work has the LPPL maintenance status `maintained'.
%% 
%% The Current Maintainer of this work is Arif Y (arifyunando@gmail.com) 
%_____________________________________________________________________________
%=============================================================================
% skripsi.tex v1_ft (17-12-2020) dibuat oleh Arif - FT UNPAR 
% 
% Ini adalah file utama (skripsi.tex), berisi perintah-perintah yang khusus 
% dibuat untuk template ini
%
% 			JANGAN MENGUBAH APAPUN DI DALAM FILE INI, 
%			KECUALI ANDA TAHU APA YANG ANDA LAKUKAN !!! 
% 
% Perubahan pada versi 13:
%   - penambahan opsi digital signature untuk para dosen
%   - penambahan tiga opsi tanda tangan untuk mahasiswa
%   - penambahan file ttd.jpg (dari Husnul Hakim)
%   - dapat menangani format npm baru (per angkatan 2018)
%   - penambahan daftar listings (kode program)
%   - penambahan gambar kotak untuk kode program
%   - pemindahan file compj.bst dan digsig.sty ke folder .sty
%   - format kode program di bab dan penambahan kotak untuk kode program
%   - penambahan daftar notasi
%
% Perubahan pada versi 13_ft:
%   - Perubahan format judul menjadi format FT TS
%   - Perubahan format penulisan abstrak
%   - Menghilangkan bagian persembahan
%   - Perubahan format penomoran halaman bab, daftar pustaka, dan lampiran
%   - Perubahan bentuk heading
%_____________________________________________________________________________
%=============================================================================

%PREAMBULE

\documentclass[11pt,a4paper,oneside,openright,notitlepage]{report}  
\usepackage[bahasa]{babel}%bahasa indonesia
%\usepackage{lmodern}   %font latin modern
\usepackage{mathptmx}   %font Times Roman
\usepackage[T1]{fontenc}%encoding
\usepackage[inner=4cm,outer=3cm,top=3cm,bottom=3cm]{geometry}%margin
\usepackage{microtype}  %mencegah underfull
\usepackage{lipsum}     %untuk dummy text
\usepackage{abstract}   %manipulasi abstract
\usepackage{tocloft}	  %format daftar isi 
\usepackage{tocbibind}  %menambah kata pengantar, daftar isi, gambar, tabel ke daftasr isi
\usepackage{fancyhdr}	  %mengubah tampilan header & footer
\usepackage{titlesec}	  %mengubah tamnpilan judul bab
\usepackage{float} 	  %penempatan gambar di tempat yg seharusnya 
\usepackage[plainpages=false,pdfpagelabels,unicode]{hyperref}% untuk \autoref,\phantomsection & link 
\usepackage{setspace}     %line spacing
\usepackage[pagewise]{lineno}%line numbering
\usepackage{emptypage}    %halaman kosong antar bab
\usepackage{graphicx}	  %untuk graphicspath
\usepackage{listings}	  %untuk penulisan source code
\usepackage{etoolbox}	  %untuk programming dll
\usepackage{xstring}	  %manipulasi string, strchar, ifstreq
\usepackage[table]{xcolor}%untuk kode program
\usepackage{pgffor}		  %untuk foreach
\usepackage{.sty/digsig}  %untuk digital signature
\usepackage{nomencl} 	  %daftar simbol

%***Untuk Pakai Natbib***
%\usepackage{natbib}
%\bibliographystyle{plainnat}

%****Untuk pakai Biblatex****
%\usepackage[backend=bibtex, style=authoryear-icomp, %sorting=nyt]{biblatex}
%	\addbibresource{referensi}	
%\providetoggle{blx@lang@captions@english}
%\providetoggle{blx@lang@captions@bahasa}

\newcommand{\vtemplateauthor}{lionov}
\newcommand{\dtext}[1]{{\color{lightgray} \lipsum[#1]}}	%command untuk dummy text
\newcommand{\rtext}[1]{{\color{red} \small #1}}			%command untuk default text

\newcommand{\vmode}{template} 						 	%default mode
\newtoggle{cekNPM}	 
%\toggletrue{cekNPM} 									%aktifkan untuk cek NPM
\newtoggle{develop}	
%\toggletrue{develop} 									%aktifkan untuk mode develop

\graphicspath{{./Gambar/}}								% folder tempat gambar 

%cek apakah text default
\newcommand{\vtext}[3]{\IfStrEq{\vmode}{template}{\newcommand{#1}{\rtext{#2}#3}}{\newcommand{#1}{#2}}}
\newcommand{\vtextup}[3]{\IfStrEq{\vmode}{template}{\newcommand{#1}{\uppercase{\rtext{#2}}#3}}{\newcommand{#1}{\uppercase{#2}}}}

%default data prodi
\newcommand{\vstaINA}{SKRIPSI} 
\newcommand{\vstaENG}{UNDERGRADUATE THESIS}
\newcommand{\vprodiINA}{TEKNIK SIPIL}
\newcommand{\vprodiENG}{CIVIL ENGINEERING}
\newcommand{\vkaprodi}{<<Ketua Program Studi>>}

%_____________________________________________________________________________
%=============================================================================
% dosen.tex v1_ft (16-12-2020) dibuat oleh Arif - Teknik Sipil FT UNPAR
%_____________________________________________________________________________
%=============================================================================

%=============================================================================
% JANGAN MENGUBAH APAPUN DI BAGIAN INI, KECUALI
% untuk mengubah kaprodi (jika kaprodi telah berganti orang) atau 
% menambahkan pembimbing anda yang tidak/belum tercantum pada daftar ini atau 
% memperbaiki penulisan gelar jika pembimbing/penguji anda meminta
%=============================================================================


%\kajur kaprodi
\renewcommand{\vkaprodi}{\FLX}

% Dosen-dosen KBI Struktur
\newcommand{\CCL}{Dr.~Cecilia~Lauw~Giok~Swan}
\newcommand{\PKW}{Dr.~Paulus~Karta~Wijaya}
\newcommand{\JON}{Dr.~Djoni~Simanta}
\newcommand{\JAT}{Dr.~Johannes~Adhijoso~Tjondro}
\newcommand{\BBS}{Prof.~Bambang~Suryoatmono,~Ph.D.}
\newcommand{\BUN}{Buen~Sian,~Ir.,~M.T.}
\newcommand{\NNS}{Nenny~Samudra,~Ir.,~M.T.}
\newcommand{\LID}{Lidya~Fransisca~Tjong,~Ir.,~M.T.}
\newcommand{\DIN}{Dr.~–Ing.~Dina~Rubiana~Widarda}
\newcommand{\HHT}{Helmy~Hermawan~Tjahjanto,~Ph.D.}
\newcommand{\HER}{Herry~Suryadi,~Ph.D.}
\newcommand{\ATS}{Altho~Sagara,~S.T.,~M.T.}
\newcommand{\SNR}{Sisi~Nova~Rizkiani,~S.T.~M.T.}

% Dosen-dosen KBI Geoteknik
\newcommand{\PPR}{Prof.~Paulus~Pramono~R.,~Ph.D.}
\newcommand{\SIS}{Siska~Rustiani,~Ir.,~M.T.}
\newcommand{\EYK}{Ng~Yin~Kuan,~Ir.,~M.T.}
\newcommand{\ASL}{Anastasia~Sri~Lestari,~Ir.,~M.T.}
\newcommand{\BUD}{Budijanto~Widjaja,~Ph.D.}
\newcommand{\ASW}{Aswin~Lim,~Ph.D.}

% Dosen-dosen KBI Sumber Daya Air
\newcommand{\TRI}{Prof.~R.~Wahyudi~Triweko,~Ph.D.}
\newcommand{\BAR}{Bambang~Adi~Riyanto,~Ir.,~M.Eng.}
\newcommand{\GOZ}{Salahudin~Gozali,~Ph.D.}
\newcommand{\YIN}{F.~Yiniarti~Eka~Kumala,~Ir.,~Dipl.HE.}
\newcommand{\DDY}{Doddi~Yudianto,~Ph.D.}

% Dosen-dosen KBI Transportasi
\newcommand{\WIM}{Prof.~Wimpy~Santosa,~Ph.D.}
\newcommand{\ATJ}{Aloysius~Tjan,~Ph.D.}
\newcommand{\ACS}{Anastasia~Caroline~Sutandi,~Ph.D.}
\newcommand{\TBJ}{Tri~Basuki~Joewono,~Ph.D.}

% Dosen-dosen KBI Manajemen Rekayasa Konstruksi
\newcommand{\ZUL}{Zulkifli~Bachtiar~Sitompul,~Ir.,~MSIE.}
\newcommand{\IWN}{A.~Tjia~Iwan~Adinata~Irawan,~Ir.,~M.T.}
\newcommand{\AAS}{Dr.~A.~Anton~Soekiman}
\newcommand{\ADI}{Yohanes~Lim~Dwi~Adianto,~Ir.,~M.T.}
\newcommand{\ROY}{Andreas~Franskie~Van~Roy,~Ph.D.}
\newcommand{\FLX}{Dr.~Felix~Hidayat}
\newcommand{\HRN}{Theresita~Herni~Setiawan,~Ir.,~M.T.}
\newcommand{\ADR}{Adrian~Firdaus,~S.T.,~M.Sc.}
\newcommand{\MIA}{Dr.Eng.~Mia~Wimala}%dosen.tex

%data mahasiswi/a dan prodi 
\newcommand{\namanpm}[2]{%
	\vtext{\vnama}{#1}{}
	\vtext{\vnpm}{#2}{}
	\newtoggle{okNPM}
	\iftoggle{cekNPM}{%cek penulisan NPM
		\newtoggle{myint} \newtoggle{mylen} \StrLen{\vnpm}[\mystrlen]
		\IfInteger{\vnpm}{\toggletrue{myint}}{}
		\IfStrEq{\mystrlen}{10}{\toggletrue{mylen}}{}
		\ifboolexpr{togl{myint} and togl{mylen}}{\toggletrue{okNPM}}{} 
	}{ 
		\nottoggle{okNPM}{
			\hypersetup{pdfauthor={#2 - #1}}
			\hypersetup{pdfsubject={\vprodiINA}}
		}{}
	}
}

%data tanggal sidang
\newcommand{\tanggal}[3]{
	\newcommand{\vtanggal}{\rtext{#1}} \IfInteger{#1}{\renewcommand{\vtanggal}{#1}}{}
	\newcommand{\vtahun}{\rtext{#3}} \IfInteger{#3}{\renewcommand{\vtahun}{#3}}{}
	\newcommand{\vbulan}{\rtext{#2}}
	\IfStrEqCase{#2}{%ubah angka bulan menjadi kata
		{0}{\renewcommand{\vbulan}{<<bulan>>}}
		{1}{\renewcommand{\vbulan}{Januari}}	{2}{\renewcommand{vbulan}{Februari}}	
		{3}{\renewcommand{\vbulan}{Maret}}	{4}{\renewcommand{vbulan}{April}}	
		{5}{\renewcommand{\vbulan}{Mei}}		{6}{\renewcommand{vbulan}{Juni}}		
		{7}{\renewcommand{\vbulan}{Juli}}	{8}{\renewcommand{vbulan}{Agustus}}	
		{9}{\renewcommand{\vbulan}{September}}	{10}{\renewcommand{vbulan}{Oktober}}
		{11}{\renewcommand{\vbulan}{November}}	{12}{\renewcommand{vbulan}{Desember}}
	[]}
}

% data primer skripsi	 
\newcommand{\judulINA}[1]{\vtextup{\vjudulINA}{#1}{} \hypersetup{pdftitle={#1}}}	
%\newcommand{\judulINA}[1]{\vtextup{\vjudulINA}{#1}{}}	
\newcommand{\judulENG}[1]{\vtextup{\vjudulENG}{#1}{}}	
\newcommand{\abstrakINA}[1]{\vtext{\vabstrakINA}{#1}{\\\dtext{2}}} 
\newcommand{\abstrakENG}[1]{\vtext{\vabstrakENG}{#1}{\\\dtext{2}}} 
\newcommand{\kunciINA}[1]{\vtext{\vkunciINA}{#1}{}\hypersetup{pdfkeywords={#1}}}
%\newcommand{\kunciINA}[1]{\vtext{\vkunciINA}{#1}{}}  
\newcommand{\kunciENG}[1]{\vtext{\vkunciENG}{#1}{}} 
\newcommand{\untuk}[1]{\vtext{\vuntuk}{#1}{}}
\newcommand{\prakata}[1]{\vtext{\vprakata}{#1}{\\\dtext{3-4}}}

%data pembimbing dan penguji	
\newcommand{\vnpemb}{2}
\newcommand{\pembimbing}[2]{\vtext{\vpembu}{#1}{} \vtext{\vpembs}{#2}{}}
\newcommand{\penguji}[2]{\vtext{\vpengi}{#1}{} \vtext{\vpengii}{#2}{}}

%settings
\newtoggle{vbimbingan}	 
\newtoggle{vsidang}	 
\newtoggle{vsidangakhir}	
\newtoggle{vfinal}	
\newcommand{\daftarIsiError}[1]{\newcommand{\vdierror}{\IfStrEq{#1}{0}{\hfil}{\hfill}}}  
\newcommand{\mode}[1]{\renewcommand{\vmode}{#1} 
	\IfStrEq{#1}{final}{\toggletrue{vfinal}}
	{\IfStrEq{#1}{sidangakhir}{\toggletrue{vsidangakhir}}
	{\IfStrEq{#1}{sidang}{\toggletrue{vsidang}}{\toggletrue{vbimbingan}}}}}
\newcommand{\linenumber}[1]{\newcommand{\vlinenum}{#1}}
\newcommand{\linespacing}[1]{\newcommand{\vspacing}{#1}} 
\newcommand{\gambar}[1]{\newcommand{\vgambar}{#1}}
\newcommand{\tabel}[1]{\newcommand{\vtabel}{#1}}
\newcommand{\kode}[1]{\newcommand{\vkode}{#1}}
\newcommand{\notasi}[1]{\newcommand{\vnotasi}{#1}}
\edef\vbab{1,2,3,4,5,6,7,8,9}
\newcommand{\bab}[1]{\IfStrEq{#1}{all}{}{\IfStrEq{\vmode}{bimbingan}{\edef\vbab{#1}}{}}}
\edef\vlamp{L1,L2,L3,L4,L5,L6,L7,L8,L9}
\newcommand{\lampiran}[1]{\IfStrEq{\vmode}{bimbingan}{\IfStrEq{#1}{all}{}
		{\IfStrEq{#1}{none}{\edef\vlamp{}}{\edef\vlamp{#1}}}}{}}
\newcommand{\ttd}[1]{\newcommand{\vttd}{#1}}
\newcommand{\ttddosen}[1]{\newcommand{\vttddosen}{#1}}

%spacing, mode buku selalu didahulukan
\newcommand{\appspacing}{
	\IfStrEqCase{\vmode}{{sidang}{\onehalfspacing}{sidangakhir}{\onehalfspacing}{final}{\singlespacing}
		{bimbingan}{\IfStrEqCase{\vspacing}{{single}{\singlespacing}{onehalf}{\onehalfspacing}{double}{\doublespacing}}}}}	

%line number
\newcommand{\vlineyes}{\linenumbers \def\linenumberfont{\normalfont\tiny\sffamily}}
\newcommand{\appline}{\IfStrEqCase{\vmode}{{sidang}{\vlineyes}{sidangakhir}{\vlineyes}{bimbingan}{\IfStrEq{\vlinenum}{yes}{\vlineyes}{}}[]}}
 
%header
\renewcommand\headrule{}
%\setlength{\headheight}{17.25pt}
\fancyhead[RE,RO]{\thepage}
\fancyhead[LE,LO]{\textsc{}}
%\fancyhead[RE]{\small{\textsc{\nouppercase{\leftmark}}}}
%\fancyhead[LO]{\small{\textsc{\nouppercase{\rightmark}}}}
\fancyfoot{}

%untuk url dan link
\hypersetup{unicode=true,colorlinks=true,linkcolor=blue,citecolor=green,filecolor=magenta, urlcolor=cyan}

%untuk penulisan judul bab
\titleformat{\chapter}[display] {\Large\bfseries\centering}{\MakeUppercase{\chaptertitlename} \thechapter}{10pt}{\Large\MakeUppercase}

%tulisan abstrak uppercase
\renewcommand{\abstractnamefont}{\bf \MakeUppercase}

%font
\normalfont %aktivasi default font
%\DeclareFontShape{T1}{lmr}{bx}{sc}{<->ssub*cmr/bx/sc}{} %scschape comp modern untuk latin modern

%setting daftar isi dan daftar lainnya
\cftsetpnumwidth{25pt}	% perbaikan untuk halaman daftar isi romawi yg overfull
\renewcommand{\cftchapfont}{\scshape \bfseries} 
\renewcommand{\cfttoctitlefont}{\vdierror\Large\bfseries\MakeUppercase}
\renewcommand{\cftaftertoctitle}{\hfill}
\renewcommand{\cftloftitlefont}{\hfill\Large\bfseries\MakeUppercase} 
\renewcommand{\cftafterloftitle}{\hfill}
\renewcommand{\cftlottitlefont}{\hfill\Large\bfseries\MakeUppercase}
\renewcommand{\cftafterlottitle}{\hfill}
\renewcommand{\lstlistingname}{Kode}
\renewcommand{\lstlistlistingname}{Daftar Kode Program}
\makenomenclature
\renewcommand{\nomname}{Daftar Notasi}

\let\Chapter\chapter
\def\chapter{\addtocontents{lol}{\protect\addvspace{10pt}}\Chapter}

%listing khusus untuk penulisan kode program, menggunakan font Bera Mono
\lstset{numbers=left,stepnumber=1, numbersep=5pt, frame=single,frameround={tttt},
	tabsize=4, breaklines=true, basicstyle=\fontfamily{fvm}\selectfont\tiny, 
	commentstyle=\itshape\color{gray}, keywordstyle=\bfseries\color{blue}, 
	identifierstyle=\color{black}, stringstyle=\color{orange},
	literate={-}{-}1{-\,-}{--}1
} 

%\toggletrue{develop} % aktifkan untuk mode develop
%apakah dalam develop mode ?
\iftoggle{develop}{ 
	\usepackage{booktabs,pgfplots}\usepackage[table]{xcolor}
	\daftarIsiError{0} 
	\mode{final} 
	\linenumber{yes}\linespacing{double} 
	\gambar{yes}\tabel{no}\kode{yes} 
	\bab{1,2}\lampiran{A} 
	\namanpm{Arif Yunando Sunanhadikusuma}{2017410211}\tanggal{9}{5}{2020}
	\pembimbing{\ASW}{}
	\judulINA{Topi Saya Bundar}
	\judulENG{My Round Hat}
	\abstrakINA{\lipsum[6]}
	\abstrakENG{\lipsum[7]} 
	\kunciINA{Kunci, Kunci-kunci, Kata, Kata-kata, Indonesia}
	\kunciENG{Key, Keys, Word, Words, English}
	\prakata{\lipsum[1-5]}
	\pgfplotsset{compat=newest}
	\ttd{kosong}
	\ttddosen{no}
	\notasi{yes}
}

\include{data} 

%isi dokumen
\begin{document}
	
\raggedbottom


%BAGIAN COVER DAN AWAL 
 
\pagestyle{empty}%tidak ada nomor halaman
\nottoggle{vbimbingan}{ %Hanya bagian isi dan lampiran, tidak ada cover/pembuka

	\pagenumbering{roman}
 
	%cover SAMPUL
	\begin{center}
		{\Large \textbf{%
			\vstaINA\\\vspace{1.5cm}
			\vjudulINA\\\vspace{2.5cm}
			\includegraphics{Logo-UNPAR}\\\vspace{3.5cm}
			\MakeUppercase{\vnama}\\\vspace{0.1cm}
			NPM: \vnpm\\\vspace{1cm}
		\ifdefempty{\vpembs}
			{\begin{center} Pembimbing : {\vpembu}\ \\\end{center}}
			{\begin{center}
				PEMBIMBING 	  : \vpembu \\ \vspace{0.1cm} 
				KO-PEMBIMBING : \vpembs\ 
			\end{center}}
		\vfill
			UNIVERSITAS KATOLIK PARAHYANGAN\\
			FAKULTAS TEKNIK JURUSAN TEKNIK SIPIL\\
			\small (Terakredetasi Berdasarkan SK BAN-PT Nomor:1788/SK/BAN-PT/Akred/S/VII/2018)\\
			BANDUNG\\
			\MakeUppercase{\vbulan}\ \vtahun 
		}}
	\end{center}
	\newpage

\nottoggle{vsidang}{ %Hanya Bagian Isi, Lampiran, dan Cover. Tidak ada kata pengantar, lembar pengesahan, dan abstrak 
	
	%cover KERTAS
	\begin{center}
		{\Large \textbf{%
		\vstaINA\\\vspace{1.5cm}
		\vjudulINA\\\vspace{2.5cm}
		\includegraphics{Logo-UNPAR}\\\vspace{3.5cm}
		\MakeUppercase{\vnama}\\\vspace{0.1cm}
		NPM: \vnpm\\\vspace{1cm}
	\ifdefempty{\vpembs}
		{\begin{center} Pembimbing : {\vpembu}\ \\\end{center}}
		{\begin{center}
			PEMBIMBING 	  : \vpembu \\ \vspace{0.1cm} 
			KO-PEMBIMBING : \vpembs\ 
		\end{center}}
	\vfill
		UNIVERSITAS KATOLIK PARAHYANGAN\\
		FAKULTAS TEKNIK JURUSAN TEKNIK SIPIL\\
		\small (Terakredetasi Berdasarkan SK BAN-PT Nomor:1788/SK/BAN-PT/Akred/S/VII/2018)\\
		BANDUNG\\
		\MakeUppercase{\vbulan}\ \vtahun 
		}}
	\end{center}
	\newpage

\nottoggle{vsidangakhir}{ %Semua ada kecuali lembar pengesahan & pernyataan dan kata pengantar

	% Lembar pengesahan
	\begin{center}
		{\Large \textbf{%
		\vstaINA\\\vspace{1.5cm}
		\vjudulINA\\\vspace{2.5cm}
		\includegraphics{Logo-UNPAR}\\\vspace{2cm}
		\MakeUppercase{\vnama}\\\vspace{0.1cm}
		NPM: \vnpm\\\vspace{0.5cm}
		\begin{center}
			BANDUNG, \vtanggal\ \MakeUppercase{\vbulan}\ \vtahun \\ \vspace{0.5cm}
			\ifdefempty{\vpembs}
			{\begin{center} Pembimbing\\ 
			    \IfStrEq{\vttddosen}{no}{\vspace{2.5cm}}			    
			    {\begin{Form} \ \digsigfield{5cm}{2.5cm}{\vpembu}\ \end{Form}\\}
			    \vpembu\end{center}}
			{\begin{minipage}[b]{0.475\textwidth}
				 \begin{center}Pembimbing \\ 
				 \IfStrEq{\vttddosen}{no}{\vspace{2.5cm}}
				 {\begin{Form} \ \digsigfield{5cm}{2.5cm}{\vpembu}\ \end{Form}\\} 
				 \vpembu\end{center}
			 \end{minipage} \hspace{0.3cm}
			 \begin{minipage}[b]{0.475\textwidth}
				 \begin{center} Ko-Pembimbing \\
				 \IfStrEq{\vttddosen}{no}{\vspace{2.5cm}}
				 {\begin{Form} \ \digsigfield{5cm}{2.5cm}{\vpembs}\ \end{Form}\\}
				 \vpembs\end{center}
			 \end{minipage}	
			}
		\end{center}
		\vfill
		UNIVERSITAS KATOLIK PARAHYANGAN\\
		FAKULTAS TEKNIK JURUSAN TEKNIK SIPIL\\
		\small (Terakredetasi Berdasarkan SK BAN-PT Nomor:1788/SK/BAN-PT/Akred/S/VII/2018)\\
		BANDUNG\\
		\MakeUppercase{\vbulan}\ \vtahun 
		}}
	\end{center}
	\newpage
	
	% Lembar Pernyataan
	\begin{center} 
		{\Large \textbf{PERNYATAAN}\\} \vspace{1cm}
	\end{center}
	\begin{flushleft}
Saya yang bertandatangan di bawah ini, menyatakan bahwa: \\\vspace{0.2cm}
Nama Lengkap : \vnama \\\vspace{0.2cm}
NPM \hspace{1.65cm}: \vnpm \\\vspace{0.2cm}
Dengan ini menyatakan bahwa \vstaINA\ saya yang berjudul: \vjudulINA\ adalah karya ilmiah yang bebas dari plagiat. Jika kemudian hari terbukti terdapat plagiat dalam skripsi ini, maka saya bersedia menerima sanksi sesuai dengan peraturan perundang - undangan yang berlaku. \\
	\vspace{0.25cm}
	\end{flushleft}
	
	\begin{flushright}	
		Bandung,\ \vbulan\ \vtahun \\
		\IfStrEq{\vttd}{digital}
		    {\begin{Form} \ \digsigfield{5cm}{2.25cm}{\vnama}\ \end{Form}\\}
		    {\IfStrEq{\vttd}{gambar}
		        {\includegraphics[width=3.5cm,height=3.5cm,keepaspectratio]{ttd}\\}
		        {{\IfStrEq{\vttd}{meterai}
		            {\vspace{0.5cm}
        		    \begin{tabular}{|p{1.75cm}|}\hline\\Meterai\\Rp. 6000\\ \hline\end{tabular}\\\vspace{0.5cm}}
        		    {\vspace{2.5cm}}
		        }
		        }    
        	}
		\vnama \\
		\vnpm
	\end{flushright}
	 \cleardoublepage 

}{}%vsidangakhir	

	% Abstrak & Abstract
	\begin{center}
		{\textbf{%
		\vjudulINA\\\vspace{0.5cm}
		{\vnama}\\\vspace{0.1cm}
		NPM: \vnpm\\\vspace{0.5cm}
			\ifdefempty{\vpembs}
			{\begin{center} Pembimbing : {\vpembu}\ \\\end{center}}
			{\begin{center}Pembimbing : \vpembu \\ \vspace{0.1cm} 
						Ko-Pembimbing : \vpembs\ \end{center}}
		UNIVERSITAS KATOLIK PARAHYANGAN\\
		FAKULTAS TEKNIK JURUSAN TEKNIK SIPIL\\
		\small (Terakredetasi Berdasarkan SK BAN-PT Nomor:1788/SK/BAN-PT/Akred/S/VII/2018)\\
		BANDUNG\\
		\MakeUppercase{\vbulan}\ \vtahun 
		}}
	\end{center}
	\renewcommand{\abstractname}{Abstrak} 
	\begin{abstract} 
		{\noindent \vabstrakINA} \vskip 1cm
		{\noindent \textbf{Kata kunci:} \vkunciINA}
	\end{abstract}
	\newpage
		\begin{center}
		{\textbf{%
		\vjudulENG\\\vspace{0.5cm}
		{\vnama}\\\vspace{0.1cm}
		NPM: \vnpm\\\vspace{0.5cm}
			\ifdefempty{\vpembs}
			{\begin{center}Advisor : {\vpembu}\ \\\end{center}}
			{\begin{center}Advisor : \vpembu \\ \vspace{0.1cm} 
						Co-Advisor : \vpembs\ \end{center}}
		PARAHYANGAN CATHOLIC UNIVERSITY\\
		FACULTY OF ENGINEERING DEPARTMENT OF CIVIL ENGINEERING\\
		\small (Accreditated by SK BAN-PT Number:1788/SK/BAN-PT/Akred/S/VII/2018)\\
		BANDUNG\\
		\MakeUppercase{\vbulan}\ \vtahun 
		}}
	\end{center}
	\renewcommand{\abstractname}{Abstract}
	\begin{abstract}
		{\noindent \vabstrakENG} \vskip 1cm		
		{\noindent \textbf{Keywords:} \vkunciENG} 
	\end{abstract}			
	\cleardoublepage

\nottoggle{vsidangakhir}{	 
	% Kata pengantar
	\chapter*{PRAKATA}
	\label{ch:prakata}
	\addcontentsline{toc}{chapter}{Kata Pengantar}
	\vprakata \vspace{0.25cm}
	\begin{flushright}	
		Bandung,\ \vbulan\ \vtahun \\
		\IfStrEq{\vttd}{digital}
		    {\begin{Form} \ \digsigfield{5cm}{2.25cm}{\vnama}\ \end{Form}\\}
		    {\IfStrEq{\vttd}{gambar}
		        {\includegraphics[width=3.5cm,height=3.5cm,keepaspectratio]{ttd}\\}
		        {{\IfStrEq{\vttd}{meterai}
		            {\vspace{0.5cm}
        		    \begin{tabular}{|p{1.75cm}|}\hline\\Meterai\\Rp. 6000\\ \hline\end{tabular}\\\vspace{0.5cm}}
        		    {\vspace{2.5cm}}
		        }
		        }    
        	}
		\vnama \\
		\vnpm
	\end{flushright}	
	\newpage
}{}%vsidangakhir
}{}%vsidang

\tableofcontents \cleardoublepage 							% Daftar isi
\IfStrEq{\vgambar}{yes}{\listoffigures \cleardoublepage}{} 	% Daftar gambar
\IfStrEq{\vtabel}{yes}{\listoftables \cleardoublepage}{} 	% Daftar tabel  
\IfStrEq{\vkode}{yes}{
    \lstlistoflistings \addcontentsline{toc}{chapter}{Daftar Kode Program}
    \cleardoublepage}{} % Daftar kode 
\IfStrEq{\vnotasi}{yes}{
    \setlength{\nomitemsep}{-\parsep}
    \printnomenclature
    \addcontentsline{toc}{chapter}{Daftar Notasi}
    \cleardoublepage}{} % Daftar notasi 
\newcounter{savepage}
\setcounter{savepage}{\value{page}}
\cleardoublepage
}{} %vbimbingan


%BAGIAN ISI
%setting halaman bab
\pagenumbering{arabic} 	%nomor halaman
\appspacing			%apply spacing
\appline				%apply line number
\setpagewiselinenumbers 	%supaya reset ke 1 di setiap halaman
\pagestyle{fancy}		%header


%include bab
\foreach \i in \vbab {\edef\vfile{Bab/bab\i} \setcounter{page}{1} \renewcommand{\thepage}{\i\hspace{1pt}-\arabic{page}} \IfFileExists{\vfile}{\include{\vfile}}} 


%BAGIAN REFERENSI DAN LAMPIRAN
%update nama daftar referensi
\renewcommand{\bibname}{Daftar Referensi}
\pagenumbering{roman}
\setcounter{page}{\value{savepage}}
\onehalfspacing								%selalu single spacing untuk lampiran dan daftar referensi
\nolinenumbers								%tidak ada line number untuk lampiran dan daftar referensi
\bibliographystyle{.sty/compj} 				%gunakan compj 
\def\bibindent{1cm}
\makeatletter
\let\old@biblabel\@biblabel
\def\@biblabel#1{\old@biblabel{#1}\kern\bibindent}
\let\old@bibitem\bibitem
\def\bibitem#1{\old@bibitem{#1}\leavevmode\kern-\bibindent}
\renewcommand\@biblabel[1]{}
\makeatother
\bibliography{referensi}					%include 
\cleardoublepage
%***Untuk pakai Biblatex***
%\printbibliography

\appendix									%Penomoran lampiran dengan huruf  

%include lampiran
\foreach \i in \vlamp { \edef\vfile{Lampiran/lamp\i}  \setcounter{page}{1} \renewcommand{\thepage}{\i\hspace{1pt}-\arabic{page}} \IfFileExists{\vfile}{\include{\vfile}}}

\end{document}
